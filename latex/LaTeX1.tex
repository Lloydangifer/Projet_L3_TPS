\documentclass{report}


\usepackage[latin1]{inputenc} % un package
\usepackage[T1]{fontenc}      % un second package
\usepackage[francais]{babel}  % un troisi�me package
\usepackage{lipsum}
\usepackage{aeguill} % Police moins floue
\usepackage{graphicx} % Ins�rer des images 

\usepackage{titling} % image on title page

\usepackage{hyperref} % Cr�er des liens et des signets, utile pour la table des mati�res
\hypersetup{	
	colorlinks=true, %colorise les liens 
	breaklinks=true, %permet le retour � la ligne dans les liens trop longs 
	urlcolor= blue, %couleur des hyperliens 
	linkcolor= black,	%couleur des liens internes 
	citecolor=black,	%couleur des r�f�rences 
	pdftitle={Rapport de projet}, %informations apparaissant dans 
	pdfauthor={Virgil Manrique, Quentin Guillien}, %les informations du document 
	pdfsubject={TPS avec le Shine Engine}	%sous Acrobat. 
} 


%\title{\includegraphics[width=3cm]{logo-ufc.jpg}\\
%\bsc{\normalsize Universit� de Franche-Comt�}\\\vspace{14mm}
%Cr�ation d'un Third-Person Shooter avec le \bsc{Shine Engine}\\\vspace{14mm}
%\includegraphics[width=4cm]{logo_shine.png}}
%\author{
	%Virgil MANRIQUE\\
	%\and
	%Quentin GUILLIEN
%}
%\date{2015-2016}
\begin{document}

%\maketitle

\begin{titlepage}
	\centering
	\includegraphics[width=4.5cm]{logo-ufc.jpg}\par\vspace{4mm}
	{\scshape\large Universit� de Franche-Comt� \par}
	\vspace{1cm}
	{\scshape\Large Rapport de projet tutor�\par
	Licence informatique 3\up{�me} Ann�e\par}
	\vspace{1.5cm}
	{\huge\bfseries Cr�ation d'un Third-Person Shooter avec le \textsc{Shine Engine}\par}
	\vspace{9mm}
	\includegraphics[width=6cm]{logo_shine.png}\par\vspace{15mm}
	{\Large\itshape Virgil \textsc{Manrique} et Quentin \textsc{Guillien}\par}
	\vfill
	Encadrant : \par
	Sylvain \textsc{Grosdemouge}

	\vfill

% Bottom of the page
	{\large Ann�e 2015-2016\par}
\end{titlepage}



\section*{Remerciements}
\phantomsection
\addscontentsline

merci maman \bsc{mdr}

% Table des mati�res, sans num�rotation
\renewcommand{\thepage}{}
\tableofcontents
\newpage
\renewcommand{\thepage}{\arabic{page}}


% Introduction
\chapter{Introduction}
Ouais alors donc ben voil�


% Al� mdr
\chapter{chapitre 1} 
\lipsum[4] 
\section{section 1} 
\lipsum[4] 
\subsection{sous-section 1} 
\lipsum[4] 
\section{section 2} 
\lipsum[4] 
\subsection{sous-section a} 
\lipsum[4] 


\end{document}